
%% The actual author of this is Gilles Castel , a fellow KU Leuven student :) 

% His work can be found at the following link: https://castel.dev/  Please support this guy if you like this template or if you get some of his ideas/work as he explains in his site for your own 


% PACKAGES 

\usepackage[utf8]{inputenc}
\usepackage[english]{babel}
\usepackage[margin=1in]{geometry}

% nice fonts
\usepackage{fontawesome}

%% Generalize fanct header and footer
\usepackage{fancyhdr}
\usepackage{changes} %% if needed for track changes

%% For quotes
\usepackage{epigraph}
\usepackage{dirtytalk}

%You can use the titlesec package to customize chapters, sections and subsections style in an easy way.
\usepackage{titlesec}

% Figure/Table and graphic packages
\usepackage{import}
\usepackage[normalem]{ulem}
\usepackage{xifthen}
\usepackage{pdfpages}
\usepackage{transparent}
\usepackage[labelsep=period, font=small, margin=0.7cm, labelfont={bf}]{caption}
\usepackage{tikz}
\usepackage{subfig}
\graphicspath{{figures/}}
\usepackage{colortbl}
\usepackage{graphicx}
\usepackage{float}

% Fancy script capitals
\usepackage{mathrsfs}

% Math related packages
\usepackage{amsmath, amsfonts, mathtools, amsthm, amssymb}
\usepackage{bm}
% si unitx
\usepackage{siunitx}
\usepackage{units}

% Todonotes and inline notes in fancy boxes
\usepackage{todonotes}
\usepackage{tcolorbox} % more info here https://www.ctan.org/pkg/tcolorbox
\tcbuselibrary{theorems}

% no intent between paragraphs
\usepackage{parskip}

% Better package for hyperlinks
\usepackage{hyperref}

% For customized footnotes (basically used in title page)
\usepackage{footmisc}

% To add code with premade highlight
\usepackage{minted}

% To define color
\usepackage{color}
\usepackage[makeroom]{cancel}

% For Margin notes
\usepackage{marginnote}

% For box around Definition, Theorem, \ldots
\usepackage{mdframed}
\mdfsetup{skipabove=1em,skipbelow=0em}
\theoremstyle{definition}

%%%%%%%%%%%%%%%%%%%%%%%%%%%%%%%%%%%%%%%%%%%%
%% BELOW ARE NEW DEFINITION AND NEW COMMANDS
\definecolor{lightgray}{gray}{0.8}
\definecolor{lightergray}{gray}{0.92}
\newcommand\tab[1][1cm]{\hspace*{#1}}
\renewcommand{\thefootnote}{\fnsymbol{footnote}}

% adding Fig. in the beggining of a new figure
\addto\captionsenglish{\renewcommand{\figurename}{Fig.}}

% HRule used in the title
\newcommand{\HRule}{\rule{\linewidth}{0.5mm}}

% for marginal notes
\newcommand{\leftsidenote}[1]{\reversemarginpar\marginnote{\scriptsize #1}}

% These are the fancy headers
\pagestyle{fancy}
% LE: left even
% RO: right odd
% CE, CO: center even, center odd
% My name for when I print my lecture notes to use for an open book exam.
% \fancyhead[LE,RO]{Your name}

\fancyhead[RO,LE]{\nouppercase\rightmark} % Right odd,  Left even
\fancyhead[RE,LO]{}          % Right even, Left odd
\fancyhead[C]{}          % Right even, Left odd

\fancyfoot[RO,LE]{\thepage}  % Right odd,  Left even
\fancyfoot[RE,LO]{}          % Right even, Left odd
\fancyfoot[C]{}     % Center

% Make boxes breakable
\tcbuselibrary{breakable}

% Enviroments to use if needed in the notes
\newenvironment{correction}{\begin{tcolorbox}[
    arc=0mm,
    colback=white,
    colframe=green!60!black,
    title= Correction,
    fonttitle=\sffamily,
    breakable
]}{\end{tcolorbox}}
% Same as 'correction' but color of box is different
\newenvironment{note}{\begin{tcolorbox}[
    arc=0mm,
    colback=white,
    colframe=white!60!black,
    title= Note,
    fonttitle=\faCommentingO \; \sffamily,
    breakable
]}{\end{tcolorbox}}

\newtcbtheorem[auto counter]{definition}{\faBook \; Definition}%
{colback=white!5,colframe=green!35!black,fonttitle=\bfseries}{def}

\newtcbtheorem[auto counter]{exercise}{\faPencilSquareO \; Exercise}%
{colback=white!5,colframe=red!35!black,fonttitle=\bfseries}{ex}

\newtcbtheorem[auto counter]{question}{\faQuestionCircleO \; Question}%
{colback=white!5,colframe=blue!35!black,fonttitle=\bfseries}{qu}

\newtcbtheorem[auto counter]{sources}{\faLink \; Sources}%
{colback=white!5,colframe=blue!50!black,fonttitle=\bfseries}{src}

%define my date
\def\mydate{\leavevmode\hbox{\twodigits\day/\twodigits\month/\the\year}}

% Fix some stuff
% %http://tex.stackexchange.com/questions/76273/multiple-pdfs-with-page-group-included-in-a-single-page-warning
\pdfsuppresswarningpagegroup=1

% Bibliography style is set below
\bibliographystyle{apalike}

% Extras
\author{Savvas Raptis}

%% The actual author of this work can be found at the following link: https://castel.dev/  Please support this guy if you like this template or if you get some of his ideas/work as he explains in his site for your own 


